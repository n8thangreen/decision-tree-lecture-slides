\section{What and why of economic evaluation}

\begin{frame}
\frametitle{What is economic evaluation?}
	\begin{itemize}
		\item Increasing acceptance that effectiveness information is necessary but not sufficient for decision making.
		\uncover
		\item Need to \alert{explicitly} consider costs and opportunity costs of different courses of action.
		\item Economic methods can contribute to the decision making process.
		\item Offer a \alert{coherent, explicit and theoretically-based} approach to:
		\begin{itemize}
			\item Identifying, measuring and valuing resource use, costs and
outcomes.
			\item Handling uncertainty.
		\end{itemize}
	\end{itemize}
\end{frame}


\begin{frame}
\frametitle{What is economic evaluation?}
	\begin{itemize}
	\item Economic evaluation: A comparison of alternative diagnostic or treatment options in terms of their \alert{costs} and \alert{outcomes}.

		\begin{itemize}
			\item \alert{Costs ($c$)}: The value of the resources involved in providing treatment and managing side-effects, symptoms and disease-related events.
			\item \alert{Outcome ($e$)}: The health effects of the intervention.
		\end{itemize}
	\end{itemize}
\end{frame}

\begin{frame}
\frametitle{4 key components of a health economic evaluation}
\begin{itemize}
	\pause
	\item \alert{Statistical model}: Estimate population parameters.
	\pause
	\item \alert{Economic model}: Combine parameters to get averages for cost and outcomes.
	\pause
	\item \alert{Decision analysis}: Summarise model results in suitable measures. What is the `best' course of action?
	\pause
	\item \alert{Uncertainty analysis}: How does different types of imperfect knowledge affect the results?
\end{itemize}
\end{frame}


\begin{frame}
\frametitle{What is economic evaluation?}
\begin{table}
	\centering
		\begin{tabular}{l l}
		\hline
\textbf{Type} & \textbf{Outcome measure} \\
\hline \hline
Cost-\alert{consequence} analysis
(CCA) & Multiple outcomes reported in\\
&disaggregated manner\\
Cost-\alert{minimisation} analysis
(CMA) & None (evidence or assumption of\\
&equivalent outcomes)\\
Cost-\alert{effectiveness} analysis
(CEA) & Natural units (e.g. life years,\\
 &cases detected)\\
Cost-\alert{utility} analysis (CUA) & QALYs (longevity \\
&and quality of life)\\
Cost-\alert{benefit} analysis
(CBA) & Monetary valuation\\
\hline
		\end{tabular}
\end{table}
\end{frame}


\begin{frame}
\frametitle{What is economic evaluation?}
	\begin{itemize}
			\item \alert{Comparative methodology}: Interested in incremental costs and outcomes- multivariate results.\\
			Can summarise population economic averages in a `cost per outcome' ratio using
	\end{itemize}
	
	\begin{definition}
Incremental Cost-Effectiveness Ratio (ICER)
$$
\frac{\Delta C}{\Delta E} = \frac{C_{new} - C_{old}}{E_{new} - E_{old}}
$$
where $C=\mathbb{E}[c|\theta]$ and $E=\mathbb{E}[e|\theta]$.
\end{definition}
	
\end{frame}


\subsection{RCTs}

\begin{frame}
\frametitle{Where can we get good evidence about cost-effectiveness?}
\begin{itemize}
\item A good source is \alert{randomised controlled trials (RCT)}:
	\begin{itemize}
		\item Unbiased estimates of treatment effects
		\item Can collect outcome and resource use information prospectively
		\item Obtain patient-specific data
	\end{itemize}
\item So why use a model to conduct an economic analysis then?
	\begin{itemize}
		\pause
		\item Patients may not be representative
		\pause
		\item May be unsuitable for population interventions
		\pause
		\item May not compare all the relevant alternatives
		\pause
		\item Trial duration may not be long enough
		\pause
		\item We may be interested in long-term/lifetime costs and effects
	\end{itemize}
\end{itemize}
\end{frame}

\begin{frame}
\frametitle{RCTs: Patients may not be representative}

	\begin{itemize}
		\item Trials tend to provide evidence \alert{specific} to a particular setting or group of patients, and this may not represent patients commonly seen in \alert{clinical practice} or reflect the requirements for the particular decision problem being posed.
		\item If there is a need to generalise to other settings or patient sub-groups, additional modelling of the trial baseline risks and resource usage informed by other sources may be required to make the results \alert{generalisable}.
\end{itemize}
\end{frame}


\begin{frame}
\frametitle{RCTs might not compare all the relevant alternatives}

	\begin{itemize}
		\item Economic evaluation is a comparative methodology for assessing the value of one course of action compared to another (or range of options).
		\item A randomized trial may provide evidence on two or three options, but is unlikely to be able to provide evidence on \alert{all the relevant options} available.
\end{itemize}
\end{frame}


\begin{frame}
\frametitle{Information in studies may have to be combined}
	\begin{itemize}
		\item A single trial is unlikely to provide all the information required, and it might be necessary to combine evidence from a \alert{range of sources}.
		\item Important to scrutinise evidence for its applicability to the evaluation being undertaken.
		\item In the case of economic evaluation this means evidence on \alert{resource utilisation, unit costs, effectiveness and quality of life}.
		\item The range of sources may include trials but also \alert{cohort studies, surveys or patient records, expert opinion}.
		\item Decision models can provide an organizing framework within which these different types of data can be synthesised.
\end{itemize}
\end{frame}


\begin{frame}
\frametitle{RCTs might not encompass time horizon}
	\begin{itemize}
		\item The appropriate time period for the purpose of an economic evaluation is the time period that is long enough to capture in full the differences in resource use, unit costs and benefits between the alternative options being evaluated.
		\item Often, as is the case for interventions for chronic disease, this requires a time horizon that captures the patients lifetime.
		\item Trials rarely provide evidence over the lifetime of all patients (except in cases of interventions for terminal illness).
		\item There is therefore a need to extrapolate beyond the trial evidence, and decision models can provide a vehicle to extrapolate evidence from trials to a longer, more appropriate, time horizon.
\end{itemize}
\end{frame}


\begin{frame}
\frametitle{Decision analysis}
Decision analysis (DA) is an \alert{explicit quantitative} approach to decision making \alert{under uncertainty}.
	\begin{itemize}
		\item Mathematical representation of a series of possible events that flow from a set of alternative options being evaluated.
		\item DA compares at least two alternatives.
		\item The likelihood of each event is expressed as probability.
		\item Each event has associated values/outcomes.
		\item DA is based on the concept of \alert{expected value (EV)}.
				\begin{itemize}
		\item For a given option, EV is the sum of the values of each event weighted by the probability of the event.
		\end{itemize}
	\end{itemize}
\end{frame}
