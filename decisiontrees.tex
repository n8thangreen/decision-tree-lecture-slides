\section{Decision Trees}

\begin{frame}
\frametitle{Steps in constructing and analysing Decision Trees}
A decision tree is a visual representation of a decision analysis:
	\begin{itemize}
		\item \alert{Structure} the tree
		\pause
		\item Estimate \alert{probabilities}
		\pause
		\item Estimate \alert{payoffs} (assign values to costs and outcomes)
		\pause
		\item Analyse the tree
		\pause
			\begin{itemize}
				\item \alert{Evaluate} the tree
				\item Explore \alert{uncertainty}
			\end{itemize}
	\end{itemize}
\end{frame}

\begin{frame}
\frametitle{Structuring the Decision Tree}
A decision tree is made up of nodes, branches and outcomes

	\begin{itemize}
		\item Nodes:
	\begin{itemize}
		\pause
		\item \textcolor{blue}{Decision node (square)}: Describes the problem. Deterministic choice.
		\pause
		\item \textcolor{green}{Chance node (circle)}: Represents the point at which several possible events can occur.
		\pause
		\item \textcolor{red}{Terminal node (triangle)}: Represents the end of a tree with a payoff attached.
	\end{itemize}
		\end{itemize}
\end{frame}


	\begin{frame}
\frametitle{Structuring the Decision Tree}
	\begin{itemize}
		\item Branches issuing from a chance node represent possible events patients may experience at that point in the tree.
		\item Branch probabilities represent the likelihood of each event.
		\item The sequence of chance nodes from left to right usually follows the sequence of events.
		\item The events stemming from a chance node must be \alert{mutually exclusive} and probabilities should sum to 1.
	\end{itemize}
\end{frame}




\begin{frame}
\frametitle{Advantages of Decision Trees}

	\begin{itemize}
		\item They enable the economic question to be structured in a \alert{meaningful} and \alert{visual} manner.
		\pause
		\item  They allow data informing the model parameters to be assimilated and, where appropriate, synthesised.
		\pause
		\item They are relatively simple to undertake and suitable for:
		\begin{itemize}
			\item Diseases that occur only once.
			\item Decisions about acute care.
			\item Decisions with short time frames.
		\end{itemize}
	\end{itemize}
\end{frame}


\begin{frame}
\frametitle{Limitations of Decision Trees}

	\begin{itemize}
		\item They do not explicitly account for passage of time:
			\begin{itemize}
				\item Passage of time accounted for by outcome measure.
				\item Limited ability to account for long term outcomes.
			\end{itemize}
			\pause
		\item Possible to add branches but results in a complex model.
		\item Other modelling techniques can handle repeated events better.
		\item Structure of tree only allows for one-way progression of patient through model: Not movement back and forth between states.
		\item Decision trees can still be useful as a sub-model.
		\end{itemize}
\end{frame}


\begin{frame}
\frametitle{}
 \includegraphics[width=1\textwidth,height=1.2\textheight, keepaspectratio]{newBMJscreenshot}
\end{frame} 

\begin{frame}
\frametitle{NICE suspected TB pathway}
 \includegraphics[width=1\textwidth,height=0.8\textheight, keepaspectratio]{NICEpathway}
\end{frame} 

\begin{frame}
\frametitle{Simple decision tree example: New TST guidelines}
 \includegraphics[width=\textwidth,height=0.8\textheight, keepaspectratio]{TSTdecisiontree}
\end{frame}

\begin{frame}
\frametitle{Analysing Decision Trees}
	\begin{itemize}
		\item The decision tree is averaged-out and `folded-back' to get the expected payoffs for each strategy.
		\item Estimated separately as the sum of products of the probability of events and their payoffs i.e.
\alert{weighted average} of the outcome values.
		\item Cost-effectiveness analysis: Strongly and extendedly dominated strategies removed and ICERs estimated.
	\end{itemize}
\end{frame}

\begin{frame}
\frametitle{Combined probabilities for each branch}
 If probability of BCG is $0.1$ (c.f. NHS Immunisation Statistics, England 2012-13) then
\begin{eqnarray*}
p_1 &=& 0.1 \times 0.1 \times 0.8 = 0.008\\
p_2 &=& 0.1 \times 0.1 \times 0.2 = 0.002\\
p_3 &=& 0.1 \times 0.9 \times 0.15 = 0.0135\\
p_4 &=& 0.1 \times 0.9 \times 0.85 = 0.0765\\
p_5 &=& 0.9 \times 0.1 \times 0.95 = 0.0855\\
p_6 &=& 0.9 \times 0.1 \times 0.05 = 0.0045\\
p_7 &=& 0.9 \times 0.9 \times 0.35 = 0.2835\\
p_8 &=& 0.9 \times 0.9 \times 0.65 = 0.5265
\end{eqnarray*}
\end{frame}

\begin{frame}
\frametitle{Expected cost and health impact}
	\begin{itemize}
		\item Cost:
		\end{itemize}
\begin{eqnarray*}
&& 0.008 \times 60 + 0.003 \times 10 + 0.0135 \times 60 + 0.0765 \times 10\\
&& + 0.0855 \times 70 + 0.0045 \times 20 + 0.2835 \times 70 + 0.5265 \times 20\\
&& = \alert{38.535}
\end{eqnarray*}
	\begin{itemize}
		\item Health:
		\end{itemize}
\begin{eqnarray*}
&& 0.008 \times 50 + 0.003 \times (-100) + 0.0135 \times (-10) + 0.0765 \times 0\\
&&  + 0.0855 \times 50 + 0.0045 \times (-100) + 0.2835 \times (-10) + 0.5265 \times 0\\
&& = \alert{0.955}
\end{eqnarray*}
\end{frame}


\begin{frame}
\frametitle{}
 \includegraphics[width=\textwidth,height=1\textheight, keepaspectratio]{cost-effectiveness-plane}
\end{frame}


\begin{frame}
\frametitle{}
 \includegraphics[width=\textwidth,height=1\textheight, keepaspectratio]{dectree-tool.png}
\end{frame}


\begin{frame}
\frametitle{Example simulated output}
\begin{figure}
\vspace*{-1cm}
	\centering
			\includegraphics[width=0.5\textwidth,height=1\textheight, keepaspectratio]{testcost-2.png}
		\includegraphics[width=0.5\textwidth,height=1\textheight, keepaspectratio]{testcost-3.png}
\end{figure}
\end{frame}

\begin{frame}
\frametitle{Example Output}
\begin{figure}
\vspace*{-1cm}
	\centering
			\includegraphics[width=0.5\textwidth,height=1\textheight, keepaspectratio]{testcost-4.png}
		\includegraphics[width=0.5\textwidth,height=1\textheight, keepaspectratio]{testcost-5.png}
\end{figure}
\end{frame}

